% Metódy inžinierskej práce

\documentclass[10pt,twoside,slovak,a4paper]{article}

\usepackage[slovak]{babel}
%\usepackage[T1]{fontenc}
\usepackage[IL2]{fontenc} % lepšia sadzba písmena Ľ než v T1
\usepackage[utf8]{inputenc}
\usepackage{graphicx}
\usepackage{url} % príkaz \url na formátovanie URL
\usepackage{hyperref} % odkazy v texte budú aktívne (pri niektorých triedach dokumentov spôsobuje posun textu)

\usepackage{cite}
%\usepackage{times}

\pagestyle{headings}

\title{Autopilot\thanks{Semestrálny projekt v predmete Metódy inžinierskej práce, ak. rok 2021/22, vedenie: Vladimír Mlynarovič}} % meno a priezvisko vyučujúceho na cvičeniach

\author{Samuel Švec\\[2pt]
	{\small Slovenská technická univerzita v Bratislave}\\
	{\small Fakulta informatiky a informačných technológií}\\
	{\small \texttt{xsvecs@stuba.sk}}
	}

\date{\small 6.11.2021} % upravte



\begin{document}

\maketitle

\begin{abstract}
Tento článok bude o tom ako funguje autopilot v lietadlách či v lodiach a taktiež pokrok v automobilovom priemysle. Rozoberie aj aktuálne chyby v softvéroch, ktoré môžu ovplyvniť spoľahlivosť a bezpečnosť autopilota a tým aj zastaviť používanie daného softvéru.
\end{abstract}

Kľúčové slová: autopilot, chyby, história

\section{Úvod}

Autopilot nie je v dnešnej dobe nový pojem. S autopilotom sa vieme stretnúť aj v našom bežnom živote, a to napríklad v leteckej či vodnej doprave. V oblasti automobilového priemyslu však nie je dostatočne autopilot využívaný. V súčasnej dobe niekoľko popredných firiem pracuje na vytvorení dokonalého automobilového autopilota, ako napríklad Tesla. Avšak, použitie autopilota prináša aj veľa hrozieb. 

Základný problém, ktorý bol naznačený v úvode, je podrobnejšie vysvetlený v časti~\ref{AFA}.
Ďalej je to rozvinuté v časti~\ref{chyby}.


\section{Autopilot v leteckej doprave} \label{AFA}

\subsection{História autopilota v letectve} 

Autopilot je technický systém, ktorý je počas prevádzky schopný zastúpiť človeka v riadení ovládaného objektu bez ďalšej ľudskej asistencie. Historicky prvý funkčný autopilot bol zostrojený v roku 1912 Lawrencom Sperrym, predstavený však bol verejnosti prvý raz až v roku 1914 a uvedený do prevádzky v lodnej doprave.\cite{AVA} Na obr.~\ref{f:obr2} môžeme vidieť vynálezcu Lawrenca Sperryho.

\begin{figure*}[tbh]
\centering
\includegraphics[scale=0.30]{obr2.jpg}
\caption{Lawrence Sperry (naľavo) - prvý funkčný autopilot}
\label{f:obr2}
\end{figure*}

\subsection{Autopilot v lietadlách} \label{AVL}
V letectve sa často stretávame s pomenovaním AFCS.\footnote{Automatic Flight Control Systems - automatické systémy riadenia letu}Tieto systémy sú univerzálne používané v komerčnom letectve. Rozsiahle uplatnenie nachádzajú aj vo vojenskom a všeobecnom letectve, ale koncepcia voľného letu, v rámci ktorej bude v budúcnosti prebiehať plne automatický let, je v podstate zameraná na zníženie preťaženia dýchacích ciest, čo je situácia, ktorá ovplyvňuje najmä komerčné letectvo. Hoci lietadlá všeobecného letectva wm musia tiež pracovať v prostredí voľného letu, väčšina týchto lietadiel má buď iba manuálne riadiace systémy, alebo je inštalácia AFCS len základná.\cite{AVA}

Hlavným účelom použitia AFCS je do určitej miery automatizovať lietanie lietadla, aby sa znížila pracovná záťaž pilotov (zvyčajne v určitej kritickej fáze letu), aby sa zachovala bezpečnosť letu. Čoraz častejšie sa AFCS používajú aj na zlepšenie základných letových vlastností lietadla (napr. na zabezpečenie dynamickej stability, aj keď bolo lietadlo navrhnuté ako staticky nestabilné), alebo na overenie základných výkonov lietadla v niektorých atmosférických podmienkach. Na dosiahnutie plne automatického letu bude potrebné, aby sa dosiahlo niekoľko dôležitých technologických a operačných systémových vývojov, ale vždy, keď sa to podarí, výsledný plne automatický systém bude fungovať prostredníctvom už vyvinutého AFCS.\cite{AFCS}

\begin{figure*}[tbh]
\centering
\includegraphics[scale=0.9]{obr1.jpg}
\caption{Funkčná schéma autopilota}
\label{f:obr1}
\end{figure*}

\section{Autopilot v automobiloch}

V tejto kapitole bude opísaná funkcia autonómnej jazdy v plne elektrických autách značky Tesla.

\subsection{Počiatky autopilota v automobiloch značky Tesla}

Od roku 2012 sa začala vyrábať Tesla Model S, ktorej však systém Tesla Autopilot nebol nebol inštalovaný, pretože ho vtedy ešte automobilka Tesla nemala vyvinutý(ani softvér ani hardvér). Daný systém sa začal inštalovať do elektrických áut značky Tesla až od roku 2014. Staršie modely automobilky Tesla nedisponovali daným softvérom ani hardvérom - ani sa do daných áut nedá Autopilot dodatočne nainštalovať. 

Od roku 2014 sa v autách výrobcu Tesla nachádza nainštalovaný hardvér, a to kamery a radary. Hardvér sa nazýval HW AP1. Aj napriek funkčnej hardvérovej jednotke nebol systém funkčný, pretože stále nebol softvér vytvorený. Softvér bol však vytvorený až v roku 2016 Teslou v spolupráci s firmou s názvom Mobileye. Po následom nainštalovaný daného softvéru do vozidiel bol celý systém spustený. Daný systém prechádza pravidelne softvérovými aktualizáciami, ktoré zlepšujú funkcionalitu celého Autopilota.

Od roku 2016 sa postupne softvérovými aktualizáciami pridávali do Autopilota rôzne funkcie, ako napríklad adaptívny tempomat, automatické zatáčanie, Smart Summon\footnote{ovládanie automobilu aplikáciou mimo vozidla}, automatické parkovanie alebo detekcia dopravných značiek. \cite{TeslaAutopilot}

\section{Autopilot v lodnej doprave} \label{AVLD}
\section{Chyby autopilota} \label{chyby}


\bibliography{literatura}
\bibliographystyle{plain} 
\end{document}