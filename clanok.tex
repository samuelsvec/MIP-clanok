% Metódy inžinierskej práce

\documentclass[10pt,twoside,slovak,a4paper]{article}

\usepackage[slovak]{babel}
%\usepackage[T1]{fontenc}
\usepackage[IL2]{fontenc} % lepšia sadzba písmena Ľ než v T1
\usepackage[utf8]{inputenc}
\usepackage{graphicx}
\usepackage{url} % príkaz \url na formátovanie URL
\usepackage{hyperref} % odkazy v texte budú aktívne (pri niektorých triedach dokumentov spôsobuje posun textu)

\usepackage{cite}
%\usepackage{times}

\pagestyle{headings}

\title{Softvér autopilota\thanks{Semestrálny projekt v predmete Metódy inžinierskej práce, ak. rok 2021/22, vedenie: Vladimír Mlynarovič}} % meno a priezvisko vyučujúceho na cvičeniach

\author{Samuel Švec\\[2pt]
	{\small Slovenská technická univerzita v Bratislave}\\
	{\small Fakulta informatiky a informačných technológií}\\
	{\small \texttt{xsvecs@stuba.sk}}
	}

\date{\small 26.10.2021} % upravte



\begin{document}

\maketitle

\begin{abstract}
Tento článok bude o tom ako funguje autopilot v lietadlách či v lodiach a taktiež pokrok v automobilovom priemysle z hľadiska softvérovej stránky. Rozoberie aj aktuálne chyby v softvéroch, ktoré môžu ovplyvniť spoľahlivosť a bezpečnosť autopilota a tým aj zastaviť používanie daného softvéru.
\end{abstract}



\section{Úvod}
Autopilot nie je v dnešnej dobe nový pojem. S autopilotom sa vieme stretnúť aj v našom bežnom živote, a to napríklad v leteckej či vodnej doprave. V oblasti automobilového priemyslu však nie je dostatočne autopilot využívaný. V súčasnej dobe niekoľko popredných firiem pracuje na vytvorení automobilového autopilota, ako napríklad Tesla. Avšak, použitie autopilota prináša aj veľa hrozieb. Základný problém, ktorý bol naznačený v úvode, je podrobnejšie vysvetlený v časti~\ref{chyby}.
Záverečné poznámky sa nachádzajú v časti~\ref{zaver}.


\section{Ako funguje autopilot} \label{nejaka}

Autopilot je bla bla bla... Z obr.~\ref{f:sensors} je všetko jasné.  

\begin{figure*}[tbh]
\centering
\includegraphics[scale=0.1]{sensors.jpg}
%Aj text môže byť prezentovaný ako obrázok. Stane sa z neho označný plávajúci objekt. Po vytvorení diagramu zrušte znak \texttt{\%} pred príkazom \verb|\includegraphics| označte tento riadok ako komentár (tiež pomocou znaku \texttt{\%}).
\caption{Senzory autopilota na Tesle}
\label{f:sensors}
\end{figure*}


\section{Chyby autopilota} \label{chyby}

Základným problémom je teda\ldots{} Najprv sa pozrieme na nejaké vysvetlenie (časť~\ref{ina:nejake}), a potom na ešte nejaké (časť~\ref{ina:nejake}).\footnote{Niekedy môžete potrebovať aj poznámku pod čiarou.}

Môže sa zdať, že problém vlastne nejestvuje\cite{Coplien:MPD}, ale bolo dokázané, že to tak nie je~\cite{Czarnecki:Staged, Czarnecki:Progress}. Napriek tomu, aj dnes na webe narazíme na všelijaké pochybné názory\cite{PLP-Framework}. Dôležité veci možno \emph{zdôrazniť kurzívou}.


\subsection{Nejaké vysvetlenie} \label{ina:nejake}

Niekedy treba uviesť zoznam:

\begin{itemize}
\item jedna vec
\item druhá vec
	\begin{itemize}
	\item x
	\item y
	\end{itemize}
\end{itemize}

Ten istý zoznam, len číslovaný:

\begin{enumerate}
\item jedna vec
\item druhá vec
	\begin{enumerate}
	\item x
	\item y
	\end{enumerate}
\end{enumerate}


\subsection{Ešte nejaké vysvetlenie} \label{ina:este}

\paragraph{Veľmi dôležitá poznámka.}
Niekedy je potrebné nadpisom označiť odsek. Text pokračuje hneď za nadpisom.



\section{Dôležitá časť} \label{dolezita}




\section{Ešte dôležitejšia časť} \label{dolezitejsia}




\section{Záver} \label{zaver} % prípadne iný variant názvu



%\acknowledgement{Ak niekomu chcete poďakovať\ldots}


% týmto sa generuje zoznam literatúry z obsahu súboru literatura.bib podľa toho, na čo sa v článku odkazujete
\bibliography{literatura}
\bibliographystyle{plain} % prípadne alpha, abbrv alebo hociktorý iný
\end{document}